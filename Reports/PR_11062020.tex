\documentclass[11pt,letterpaper]{article}
\usepackage{fullpage}
\usepackage[top=2cm, bottom=4.5cm, left=2.5cm, right=2.5cm]{geometry}
\usepackage{amsmath,amsfonts,amssymb}
\usepackage{lastpage}
\usepackage[inline]{enumitem}
\usepackage{fancyhdr}
\usepackage{mathrsfs}
\usepackage{xcolor}
\usepackage{graphicx}
\usepackage{hyperref}
\usepackage{subcaption}
\hypersetup{colorlinks=true, linkcolor=blue, linkbordercolor={0 0 1}}

\renewcommand{\arraystretch}{1.75}

\setlength{\parindent}{0.0in}
\setlength{\parskip}{0.05in}

\newcommand{\its}{\item[\tiny\textbullet]}

\pagestyle{fancyplain}
\lhead{Brad Cownden}
\chead{}
\rhead{June 11, 2020}
\cfoot{\small\thepage}
\headsep 36pt

\begin{document}
\vspace{.2in}
\begin{center}
    {\bf GPU Solutions for PSCAD: IT17112}
\end{center}

	\vspace{.25in}

\begin{tabular}{| p{0.2\textwidth} | p{0.75\textwidth} |}
	\hline
	Reporting Period & May 28, 2020 - June 11, 2020 \\ \hline

	Activities & \begin{enumerate*}
	\item[\tiny\textbullet] Received updated \emph{Province} data that removed known issue of
  high level of difference between three lines of data. Data was processed and used
  to calculate new result vectors for the analysis below. \newline
  \its To establish the degree of accuracy in the GPU-based solving method, QRFactor can solve
  the system using either GPU-based methods or CPU-based methods. These results were compared
  to each other over all time steps and a histogram of the relative differences between the
  outputs was made. See figure~\ref{f:cpuvsgpu}. No relative differences greater than 10$^{-5}$
  were found. \newline
  \its The relative difference between the data sets $y(t)$ and $x(t)$ is defined as: $\Delta_{rel}(t) = |y(t) - x(t)| / \max(|y(t)|, |x(t)|)$. \newline
  \its Likewise, the new \emph{Province} data from the two compilers, CompilerIF15 and CompilerGF462,
  was also compared. We see that there are some relative differences between the data that
  are larger than 10$^{-1}$. \newline
  \its Finally, two types of output from QRFactor were compared against the data from the two
  compilers. The data from CompilerGF462 most closely matches both QRFactor outputs: differences
  are typically less than 10$^{-7}$, with no values of $\Delta_{rel}$ greater than 10$^{-5}$. \newline
  \its {\bf Conclusion}: the results from the GPU-based methods of QRFactor are consistent with
  the results of the GPU-based method, and with the provided output data to an acceptable degree. \newline
  \its Debugging QRFactor on U~of~W servers. Anticipate new timing data very soon.
	\end{enumerate*} \\ \hline

	Issues & \begin{enumerate*}
	\item[\tiny\textbullet] None
	\end{enumerate*} \\ \hline

	Milestones \newline Accomplished & \begin{enumerate*}
	\item[\tiny\textbullet] New \emph{Province} data processed and used with QRFactor. \newline
  \its GPU-based and CPU-based methods in QRFactor produce results consistent with known results.
  \end{enumerate*} \\ \hline

	Milestones Not \newline Accomplished & \begin{enumerate*}
	\item[\tiny\textbullet] None
	\end{enumerate*} \\ \hline

	Next Week's \newline Milestones & \begin{enumerate*}
  \item[\tiny\textbullet] Full run of new \emph{Province} data on U of W servers.
	\end{enumerate*} \\ \hline

	Forwarded Issues & \begin{enumerate*}
	\item[\tiny\textbullet] None
	\end{enumerate*} \\ \hline
\end{tabular}

\begin{figure}[!ht]
    \centering
    \includegraphics[width=\textwidth]{C:/Users/bradc/Documents/MHI/QRFactor/MHI_QRFactor_CrossCompare.pdf}
    \caption{Cross-comparing the result vectors $\mathbf{X}(t)$ from: CompilerIF15, CompilerGF462,
    QRFactor CPU method, and QRFactor GPU method. In each case, the relative difference~$\Delta_{rel}(t)$
    between the data sets was calculated for all time steps in common -- typically 298. A
    minimum threshold difference of 1.0~x~10$^{-16}$ was used.}
    \label{f:cpuvsgpu}
\end{figure}


\end{document}
